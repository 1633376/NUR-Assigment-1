\subsection*{Question 2.d \hrule}
\textbf{Question}
\begin{quote}
Now we want to generate 3D satellite positions such that they statistically follow the satellite profile in equation (2) of the assignment; that is, the probability distribution of the (relative) radii $x \in [0, x_{max})$ should be $p(x)dx = n(x)4\pi x^2 dx/ \langle N_{sat} \rangle$ (with the same a,b and c as before). Use one of the methods discussed in class to sample this distribution. Additionally, for each galaxy generate random angles $\phi$ and $\theta$ such that the resulting positions are (statistically) uniformly distributed as a function of direction. Output the positions ($r, \phi, \theta$) for 100 such satellites.
\end{quote}

\textbf{Solution} 
\begin{quote}
Let $v$ be a uniform random variable between 0 and 1. A uniform distributed $\phi$ and $\theta$ on a sphere are then given by,

\begin{align}
\phi &= 2 \pi v \\
\theta &=  \arccos(1-2v)
\end{align}

These equation can be derived from a uniform distribution for the solid angle. This derivation is not asked in the question and the equations are therefore directly taken from the first slide of the 5th lecture.

The radius $x$ is sampled from the given probability distribution $p(x) = n(x)4 \pi x^2/ \langle N_{sat} \rangle$ with the help of rejection sampling. To apply rejection sampling first a distribution $p_{enc}(x)$ is chosen that encloses all of $p(x)$ when the distribution is multiplied with a scalar $k$ (i.e $p(x) < kp_{enc}(x) \ \forall \ x \in[0,5)$). The chosen distribution is an uniform distribution for $x \in[0,5)$,
\begin{equation}
p_{enc}(x) = 
     \begin{cases}
       \frac{1}{5} &\quad\text{for $0< x < 5$} \\
       0  &\quad\text{otherwise}
     \end{cases}
\end{equation}  

The scalar is chosen to be $k = 2$. To apply rejection sampling it must first be possible to generate samples from $p_{enc}$. To sample from this distribution the fundamental transformation law of probability is applied. Let $y$ be a uniform random variable between 0 and 1.  Let $z$ be distributed according to $p_{enc}$. The fundamental transformation law of probabilities then yields, 
\begin{equation}
\int_0^z p_{enc}(z') dz' = \int_0^y 1 dy' 
\end{equation}

Solving this for $z$ gives the transformation that is needed to let $Y$ be distributed according to $p_{enc}$,
\begin{equation}
 z = 5y
 \label{eq:transf}
\end{equation}

%The variable $y$ can be generated from the random number generator and the above transformation allows us to transform it to a sample distributed according to $p_{enc}$. 

Samples distributed according to $p(x)$ can now be found with rejection sampling:  Generate a random uniform variable $y$ between 0 and 1. Transform this variable to obtain a random variable z distributed according to $p_{enc}$. If a new random variable $w$ generated\footnote{Let $u$ be an uniform variable between 0 and 1, then w is given by $w = k*p_{enc}(z)*u$. This can be derived on exactly the same way as equation \ref{eq:transf}.} from an uniform distribution between 0 and $k p_{enc}(z)$  is smaller or equal than $ p(z)$, then $z$ is a sample from the distribution of which we want to sample. If the condition does not hold reject the sample and repeat the above process.    



The code used for the sample generation and the printing of the results consists of two files. The first file is the file that prints the results: \textsf{./code/assignment2.d.py}.  The second file, \textsf{./code/mathlib/rng.py}, contains the functions \textsf{rejection\_ sample} and \textsf{gen\_ uniform\_ spherical\_ surface\_ coords}) that are used to obtain the result. \textbf{The code for the second file can be found on pages 6 and 7}. The code for the first file and its output can be found on the next page.


\end{quote}

\textbf{Code - output}
\begin{quote}

The code that generates the output. 
\lstinputlisting{./code/assignment2_d.py}
\end{quote}

\textbf{Output - text}
\begin{quote}
The generated output.
\lstinputlisting{./output/assignment2_d_out.txt}
\end{quote}











